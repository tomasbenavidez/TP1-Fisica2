\documentclass[12pt]{article}
\usepackage[utf8]{inputenc}
\usepackage[spanish]{babel}
\usepackage{graphicx}
\usepackage{amsmath}
\usepackage{amsfonts}
\usepackage{siunitx}
\usepackage{geometry}
\usepackage{fancyhdr}
\usepackage{caption}
\usepackage{subcaption}
\usepackage{float}
\usepackage{booktabs}
\usepackage{hyperref}

\geometry{a4paper, margin=2.5cm}
\pagestyle{fancy}
\fancyhf{}
\rhead{Nombre Apellido}
\lhead{Informe de Laboratorio}
\cfoot{\thepage}

\title{Título del Informe}
\author{Nombre del estudiante \\ Facultad / Carrera \\ Curso / Año}
\date{Fecha}

\begin{document}

\maketitle

\section*{Resumen}
\addcontentsline{toc}{section}{Resumen}
% Breve descripción del objetivo y resultados del experimento.

\section{Introducción}
% Fundamento teórico, leyes físicas involucradas y objetivos.

\section{Materiales y Métodos}
\subsection{Materiales}
% Lista de instrumentos y componentes utilizados.

\subsection{Procedimiento}
% Descripción paso a paso de lo realizado en el laboratorio.

\section{Resultados}
% Tablas, gráficos, mediciones y observaciones relevantes.

\section{Análisis y Discusión}
% Interpretación de los resultados, comparación con la teoría, análisis de errores.

\section{Conclusión}
% Conclusiones generales, aprendizajes y observaciones finales.

\section*{Bibliografía}
\addcontentsline{toc}{section}{Bibliografía}
% Lista de fuentes teóricas, libros, artículos, etc.

\section*{Anexos}
\addcontentsline{toc}{section}{Anexos}
% Cálculos adicionales, esquemas eléctricos, datos crudos, etc.

\end{document}
